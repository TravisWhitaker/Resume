% LaTeX resume using res.cls
\documentclass[line,margin]{res} 

\begin{document}

\name{Travis M. Whitaker}
% \address used twice to have two lines of address
\address{3241 Nathaniel Rochester Hall, Rochester, NY 14623}
\address{(315) 391-7268 - tmw4661@rit.edu}
%\address{tmw4661@rit.edu}
%\address{https://github.com/TravisWhitaker}

\begin{resume}
 
\section{EDUCATION} {\sl Bachelor of Science,} Chemical Engineering - PGPA 4.0 \\
                % \sl will be bold italic in New Century Schoolbook (or
	        % any postscript font) and just slanted in
		% Computer Modern (default) font
                Rochester Institute of Technology, Rochester, NY, 
                expected graduation: May 2016 \\
                Minor: Computer Engineering \\
                Concentration: German Language \\
\section{SKILLS} {\sl Languages \& Software:} C, Mathematica,
						\LaTeX, C++\\
                {\sl Operating Systems:} Linux, Microsoft Windows.
				{\sl Hardware:} HP/Soltec Osciliscopes, Tektronix Function Generators, various other chemical/electrical laboratory apparatus.\\
				{\sl Labs:} General and Analytical Chemistry, Organic Chemistry, Physics.

\section{EXPERIENCE} {\sl Scientist} \hfill June 2013 - Present \\
		Janssen Biotech, a Johnson\&Johnson Company,
		Philadelphia, PA
		\begin{itemize}  \itemsep -2pt
		\item   Data Warehousing and Multidimensional Chemical Process Analysis.
		\item   Development of Novel Process Control Algorithms.
		\item   Preparation of Presentations and Regulatory Reports.
		\end{itemize}

                {\sl Researcher/Laboratory Technician} \hfill Summer 2011 \\
                Applied Electrostatics Laboratory, 
                SUNY Oswego, Oswego, NY
                 \begin{itemize}  \itemsep -2pt % reduce space between items
                \item   Design, construction, and maintenance of 
				        specialized laboratory apparatus.
                \item   Design, implementation, and analysis of
				        experiments exploring potential applications
						of electrospray ionization focusing
						on near-vacuum environments.
				\item   Manufacture of thin-film mirrors in
				        near-vacuum environments for use in
						associated laboratories.
				\item   Wrote software application in C
						for tracking individual particles
						to aid the analysis of high-speed footage
						of clouds of electrostatic particles
						produced via electrospray ionization.
                \end{itemize}

\section{PROFESSIONAL \\ INTERESTS, \\ HOBBIES, AND \\ PROJECTS}
				\begin{itemize} \itemsep -2pt
				\item	Digital Signal Processing
				\item	Computer Algebra
				\item	Embedded Systems Programming
				\item	High-End Audio
				\item	Discrete Electronics
				\item	German Culture
				\end{itemize}

\section{PROFESSIONAL ORGANIZATIONS}             
            	Computer Science House\\
                American Institute of Chemical Engineering\\
				American Mathematical Society\\
				American Chemical Society\\
				Association of Computing Machinery\\


\end{resume}
\end{document}
